\documentclass{article}
\usepackage[utf8]{inputenc}
\usepackage{amsmath}
\usepackage{amssymb}
\usepackage{amsfonts}
\usepackage{tikz}
\usepackage[margin=1in,headheight=13.6pt]{geometry}
\usepackage{amsthm}
\theoremstyle{definition}
\newtheorem{definition}{Definition}[section]
\theoremstyle{thrm}
\newtheorem{thrm}{Theorem}[section]
\usepackage{graphicx}
\renewcommand{\baselinestretch}{1.5}
\usepackage{color}  
\usepackage{hyperref}
\hypersetup{
    colorlinks=true
    linktoc=all
    linkcolor=blue
}
\usepackage{fancyheadings}
\pagestyle{fancyplain}
\fancypagestyle{plain}{
\renewcommand{\headrulewidth}{0.4pt}
}
\lhead{\fancyplain{Heather Tan}{Heather Tan}}
\rhead{\fancyplain{MAT233 lecture Notes}{MAT223 Lecture Notes}}
\title{MAT233 Linear Algebra - Lecture notes}
\author{Heather Tan}
\date{Sep - Dec 2017}
\begin{document}

\maketitle	
\tableofcontents
\pagebreak

\section{System of linear equations}
E.g $ax+bx=c, ax+by+cz=d$
\theoremstyle{definition}
\begin{definition}
	A \textbf{linear equation} in n variables/unknowns $x_1,x_2,\cdots,x_n$ is a linear equation of form $a_1x_1+a_2x_2+\cdots+a_n x_n=b$
\end{definition}
\begin{itemize}
	\item $a_1,\cdots,a_n$ are called coefficients of system($a_1,\cdots,a_n \in \mathbb{R}$)
	\item $b\in\mathbb{R}$ called constant term or right-hand side.
	\item a solution to $a_1x_1+a_2x_2+\cdots+a_n x_n=b$ are numbers $s_1, s_2,\cdots,s_n$ such that $x_1=s_1,\cdots, x_n=s_n \implies a_1x_1+a_2x_2+\cdots+a_n x_n=b$ is satisfied
\end{itemize}
\begin{definition}
	A \textbf{system of m linear equations} in n unknowns $x_1,x_2,\cdots,x_n$ is of form
\end{definition}
\begin{equation}
	\begin{matrix}
	\label{**}
	a_{11}x_1 + a_{12}x_2 + \cdots + a_{1n}x_n &= b_1\\
		a_{21}x_1 + a_{22}x_2 + \cdots + a_{2n}x_n &= b_2\\
		\cdots \quad \quad \quad \quad \quad \quad \quad &\cdots\\
		a_{m1}x_1 + a_{m2}x_2 + \cdots + a_{mn}x_n &= b_m\\
\end{matrix}
\end{equation}


\begin{itemize}
	\item $a_{ij} \in \mathbb{R}, i = 1,\cdots, m, j = 1,\cdots,n$ are called the coefficients of system
	\item $b_i \in \mathbb{R}, i = 1,\cdots, m $ are constant terms/right-hand side.
\end{itemize}
E.g $a_{ij}$ is coefficient of $x_j$ in $i^{th}$ equation.
\begin{definition}
	A solution to matrix \ref{**} are numbers $s_1, s_2,\cdots,s_n$ s.t if $x_1=s_1,\cdots, x_n=s_n$ all equations in \ref{**} are satisfied simultaneously.
\end{definition}
\textbf{Goal:} To find out if a system of linear equation has a solution and if it does, how ot find all the solutions.
\begin{itemize}
	\item One solution: $x_1+x_2=1, x_1-x_2=0$
	\item Infinity solutions: $x_1+x_2=1, 3x_1+3x_2=3$
	\item No solution: $x_1+x_2=1, x_1+x_2=2$
\end{itemize}
This example illustrate(but does not proof) what can happen in general a system of linear equations can have either a unique solution, no solution or infinitely many solutions. \\
\textbf{Question}: Why is it that a system of linear equations cannot have exactly two solutions?\\
\textbf{Answer}: Generally procedure for solving system of linear equations: elimination  -- eliminate some variables from some equations to produce a simpler system that is easy to solve $\neq$  the has the same set of solutions as original system.
\begin{definition}
	two systems of linear equations that have exactly the same set of solutions are called equivalent systems.
\end{definition}
\begin{itemize}
	\item equivalent does not mean equal, it's a weaker for of equality
	\item in elimination, each step(operation) must be reversable to produce an equivalent system.
\end{itemize}
E.g Solve\\
{\center{
$
\begin{cases}
	x_1+2x_2+x_4 &= 7\\
	x_1+x_2+x_3-x_4 &= 3\\
	3x_1+x_2+5x_3-7x_4 &=-4
\end{cases}$
}
\\}
This system can be represented by an augmented matrix
\begin{equation*}
	\begin{pmatrix}
		&1 & 2 & 0 & 1 &\vline&7&\\
		&1 & 1 & 1 & -1 &\vline&3&\\
		&3 & 1 & 5 & -7 &\vline&1&\\
	\end{pmatrix}\quad \rightarrow^{R2-R1}
	\begin{pmatrix}
		&1 & 2 & 0 & 1 &\vline&7&\\
		&0 & -1 & 1 & -2 &\vline&-4&\\
		&3 & 1 & 5 & -7 &\vline&1&\\
	\end{pmatrix}\\ \quad 
\end{equation*}
\begin{equation*}
	\rightarrow^{R3-3R1} 
	\begin{pmatrix}
		&1 & 2 & 0 & 1 &\vline&7&\\
		&0 & -1 & 1 & -2 &\vline&-4&\\
		&0 & -1 & 5 & -10 &\vline&-20&\\
	\end{pmatrix}\quad \rightarrow^{R3-5R2}
	\begin{pmatrix}
		&1 & 2 & 0 & 1 &\vline&7&\\
		&0 & -1 & 1 & -2 &\vline&-4&\\
		&0 & 0 & 0 & 0 &\vline&0&\\
	\end{pmatrix}\quad
\end{equation*}
\begin{equation*}
	\rightarrow^{R1+2R2} 
	\begin{pmatrix}
		&1 & 2 & 0 & -3 &\vline&-1&\\
		&0 & -1 & 1 & -2 &\vline&-4&\\
		&0 & 0 & 0 & 0 &\vline&0&\\
	\end{pmatrix}\quad \rightarrow^{-R2}
	\begin{pmatrix}
		&1 & 2 & 0 & -3 &\vline&-1&\\
		&0 & 1 & -1 & 2 &\vline&4&\\
		&0 & 0 & 0 & 0 &\vline&0&\\
	\end{pmatrix}\quad
\end{equation*}
Then we can conclude$
\begin{cases}
	x_1 = -1-2x_3+3x_4\\
	x_2 = 4+x_3-2x_4\\
	x_3,x_4 \in \mathbb{R}
\end{cases}$
\begin{itemize}
	\item parametric representation of solutions
	\begin{equation*}
	x_3 = s, x_4 = t, \quad s,t \in \mathbb{R}\\
	x_1 = -1-2s+3t,\quad x_2 = 4+s-2t,\quad x_3 = s,\quad x_4=t
	\end{equation*}\\
	s,t called parameters
	\item check: bring the parametric representation to original equations. If there is no solution with parameter, there will be unique solution.
\end{itemize}

\pagebreak
\section{Types of Operations Allowed in Elimination}
\textbf{Elementary row operations}
\begin{enumerate}
	\item \textbf{Adding:} adding a multiple of one row to another
	\item \textbf{Multiply:}	 multiply row by non-zero constant
	\item \textbf{Interchange:} interchange any two rows
\end{enumerate}
\begin{definition}
	The $1^{st}$ non-zero entry from left in each row is called a leading entry(variable)
\end{definition}
\subsection{Row-Echelone Form(REF)}
\begin{enumerate}
	\item each leading entry is to the right of one above it
	\item any row consisting entirely of zeros is at the bottom of matrix
\end{enumerate}
\begin{definition}
	A matrix is in rref if:
	\begin{enumerate}
		\item it is in ref
		\item each leading entry is a 1
		\item each leading entry is only none-zero entry in its column
	\end{enumerate}
\end{definition}
\textbf{Summary}: Theorem: Every system of linear equations is equivalent to one in echelon form (ref or rref). An echelon form can be found by applying a sequence of elementary row operations to original system.
\pagebreak
\section{Consistent System}
\begin{definition}
	A system of linear equations that has at least one solution is called consistent; otherwise inconsistent
\end{definition}
e.g Find the solutions on a,b,c s.t:
\begin{align*}
x_1+3x_2+x_3=a\\
-x_1-2x_x+x_3=b\\
3x_1+7x_2-x_3=c	
\end{align*}
is consistent.\\
\textbf{Solution:}
\begin{equation*}
	\begin{pmatrix}
	& 1 & 3 & 1 &\vline &a &\\
	& -1 & 2 & 1 &\vline &b &\\
	& 3 & 7 & -1 &\vline &c &
	\end{pmatrix}\quad \rightarrow^{R_2+R_1}_{R_3-3R1}
	\begin{pmatrix}
	& 1 & 3 & 1 &\vline & a &\\
	& 0 & 1 & 2 &\vline & a+b &\\
	& 0 & -2 & 4 &\vline & c-3a &
	\end{pmatrix} \quad
\end{equation*}
\begin{equation*}
\rightarrow^{R_3+2R_2}
	\begin{pmatrix}
	& 1 & 3 & 1 &\vline &a &\\
	& -1 & 2 & 1 &\vline &b &\\
	& 3 & 7 & -1 &\vline &c &
	\end{pmatrix}\quad \implies \text{System is consistent provided $c-3a+2a+2b=c-a+2b=0$}
\end{equation*}
\theoremstyle{thrm}
\begin{thrm}
	Supposed augmented matrix of system of linear equation is in ref(or rref)
	\begin{enumerate}
		\item system is consistent iff ref of augmented matrix of system has row of form $(0\quad 0 \cdots 0 \quad \vline \quad c) $ given $c \neq 0$
		\item if system is consistent $\exists$ 2 possibilities
		\begin{enumerate}
			\item number of leading variables = variables in system and system has a unique solution
			\item number of leading variables $<$ number of variables in system and system has infinitely many solutions
		\end{enumerate}	
	\end{enumerate}
\end{thrm}
\textbf{Proof}:
\begin{enumerate}
	\item if $\exists$ row in the form $(\quad 0 \quad 0 \cdots 0 \quad \vline \quad c\quad ), c\neq 0$ then this coresponds to equation $0=c, c\neq 0$ which has no solution
	\item assume system is consist i.e. $(\quad 0 \quad 0 \cdots 0 \quad \vline \quad c\quad ), c =0$ or each of rows have a leading variables which can be solved by assigning values to non-leading variables and so determind value of leading variables
	\begin{enumerate}
		\item if number of variables  =  number of leading variables, then there does not exist any free(non-leading) variables, and system has unique solution
		\item if number of leading variables $<$ number of variables in system, then $\exists$ free(non-leading) variables which are assigned as parameters in solution $\implies \exists$ infinitely many solutions 
	\end{enumerate}
\end{enumerate}

\pagebreak
\section{Vectors}

\begin{align*}
	\mathbb{R}^n&= \{(x_1,\cdots,x_n)^T \quad |\quad x_1,\cdots ,x_n \in \mathbb{R} \}\\
	& = \{(x_1,\cdots,x_n) \quad|\quad x_1,\cdots ,x_n \in \mathbb{R} \}\\
	& =\text{set of ordered n-tuples of real numbers}
\end{align*}
\begin{itemize}
	\item point(a,b,c) horizontal
	\item vector $(a,b,c)^T$ vertical
	\item $\exists$ 1-1 corresponding btw points(a,b,c) and vectors 
	\item point(a,b,c) determines a unique vectors (vector from (0,0,0) to (a,b,c))
\end{itemize}
\subsection{Basic operations with vectors}
\begin{enumerate}
	\item equality: $x,y \in \mathbb{R}^n, x=y \Longleftrightarrow x_i = y_i, i = 1,\cdots n $.
	\item vector addition $x,y \in \mathbb{R}^n$ (:= means equal by definition)
	\begin{equation*}
		x+y:=
		\begin{pmatrix}
			x_1+y_1\\
			\vdots \\
			x_n+y_n
		\end{pmatrix}
	\end{equation*}
	\item scalar multiplication: $x \in \mathbb{R}^n, c\in R $
	\begin{equation*}
		cx =  
		\begin{pmatrix}
			cx_1\\
			\vdots \\
			cx_n
		\end{pmatrix}
	\end{equation*}
\end{enumerate}

\subsection{Properties of vector addition and scalar multiplication}
\begin{enumerate}
	\item (AC) Additive Closure: $\forall x, y \in \mathbb{R}^n, x+y \in \mathbb{R}^n$
	\item (C) Communitativity: $\forall x,y \in \mathbb{R}^n, x+y = y+x$
	\item (AA) Associativity of Addition: $\forall x,y ,z\in \mathbb{R}^n, (x+y)+z = x + (y+z)$
	\item (Z) Zero vector: $\forall x \in \mathbb{R}^n, \exists 0 = (0,\cdots,0)^T \in \mathbb{R}^n$ s.t x+0=0
	\item (AI) Addictive Inverse: $\forall x\in \mathbb{R}^n, \exists x \in \mathbb{R}^n$ s.t $x + (-x) = 0$
	\item (SC) Scalar Closure: $\forall x\in \mathbb{R}^n, c \in R, cx \in \mathbb{R}$
	\item (DVA) Distributivity of Vector Addition: $\forall x,y \in \mathbb{R}^n, c \in R, c(x+y) = cx+cy $
	\item (DSA) Distributivity of Scalar Addition: $\forall x\in \mathbb{R}^n, c,d \in R, (c+d)x = cx+dx $
	\item (SMA) Scalar Multiplication Associativity: $\forall x,y \in \mathbb{R}^n, c,d \in R, (cd)x = c(dx)$
	\item (O) One: $\forall x,\in \mathbb{R}^n, 1x=x$
\end{enumerate}
\begin{definition}
	$v_1, \cdots, v_k \in \mathbb{R}^n, c_1,\cdots, c_k \in R$, the vector $c_1v_1+,\cdots, +c_kv_k$ is called the \textbf{linear combination} of $v_1,\cdots, v_k$
\end{definition}
E.x: $2\begin{pmatrix}
	1\\-1\\2
\end{pmatrix} - 3\begin{pmatrix}
	4\\-5\\-7
\end{pmatrix} = \begin{pmatrix}
	-10\\13\\25
\end{pmatrix}$ is a linear combination of $\begin{pmatrix}
	1\\-1\\2
\end{pmatrix} $ and $\begin{pmatrix}
	4\\-5\\-7
\end{pmatrix}$

\begin{definition}
	$v_1,\cdots,v_k \in \mathbb{R}^n$ the set of all linear combinations of $v_1,\cdots, v_k$ is called the span of $v_1,\cdots, v_k$ \\
	span\{$v_1,\cdots, v_k$ \} = \{$c_1v_1+,\cdots, +c_kv_k | c\in \mathbb{R}$\}
\end{definition}
E.x: 
\begin{enumerate}
	\item span\{$\begin{pmatrix}
		0\\0\\0
	\end{pmatrix}$\} = \{$\begin{pmatrix}
		0\\0\\0
	\end{pmatrix}$\}
	\item span\{$\begin{pmatrix}
		1\\2\\1
	\end{pmatrix}$\} = \{c$\begin{pmatrix}
		1\\2\\1
	\end{pmatrix} | c \in R$\} is the line through the origin in $\mathbb{R}^3$ that contains point (1,2,1)
	\item span\{$\begin{pmatrix}
		1\\2\\1
	\end{pmatrix}\begin{pmatrix}
		2\\0\\7
	\end{pmatrix}$\} = \{$c_1\begin{pmatrix}
		1\\2\\1
	\end{pmatrix}+c_2\begin{pmatrix}
		2\\0\\7
	\end{pmatrix} | c_1, c_2 \in R $\} is a plane through origin containing (1,2,1), (2,0,7)
	\item span\{$\begin{pmatrix}
		1\\2\\1
	\end{pmatrix}\begin{pmatrix}
		2\\0\\7
	\end{pmatrix}$\} = span\{$\begin{pmatrix}
		1\\2\\1
	\end{pmatrix}$\}
\end{enumerate}
\textbf{Q \& A:}
\begin{enumerate}
	\item Q: Is $0\in \mathbb{R}^n \in$ span \{$x_1,\cdots, x_n$\}?\\A: Yes $0 = 0x_1+\cdots+x_n$
	\item Q: Is $x_1 \in$ span \{$x_1,\cdots, x_n$\}?\\A: Yes $x_1 = 1x_1+0x_2+\cdots+x_n$, and $x_i \in$ span \{$x_1,\cdots, x_n$\} $1\leq i \leq k$
\end{enumerate}

\pagebreak
\textbf{Key Example: }\\
Q:  Is $(-5,11,-7)^T \in $ span\{$(1,-2,2)^T,(0,5,5)^T, (2,0,8)^T$\}?\\
i.e Asking: can $(-5,11,-7)^T$ be written as linear combination of $(1,-2,2)^T,(0,5,5)^T, (2,0,8)^T$ \\
i.e Asking does $\exists$ scalars $c_1, c_2, c_3$ s.t
\begin{equation*}
	c_1\begin{pmatrix}
		1\\-2\\2
	\end{pmatrix}+c_2\begin{pmatrix}
		0\\5\\5
	\end{pmatrix}+c_3\begin{pmatrix}
		2\\0\\8
	\end{pmatrix} = \begin{pmatrix}
		-5\\11\\-7
	\end{pmatrix}
\end{equation*}
i.e Does $\exists$ scalars $c_1, c_2, c_3$ s.t
\begin{equation*}
	\begin{pmatrix}
	c_1+\quad+2c_3\\-2c_1+5c_2+\quad \\2c_1+5c_2+8c_3
	\end{pmatrix} = \begin{pmatrix}
	-5 \\ 11\\ 7
	\end{pmatrix}
\end{equation*}
i.e does the system below have a solution
\begin{align*}
	c_1+\quad+2c_3&=-5\\-2c_1+5c_2+\quad &=11\\2c_1+5c_2+8c_3&=-7
\end{align*}
\begin{equation*}
	\begin{pmatrix}
		&1 & 0 & 2 &\vline&-5&\\
		&-2 & 5 & 0&\vline&11&\\
		&2 & 5 & 8 &\vline&-7&\\
	\end{pmatrix}\quad \rightarrow_{R3-2R1}^{R2+2R1}
	\begin{pmatrix}
		&1 & 0 & 2 &\vline&-5&\\
		&0 & 5 & 4 &\vline&1&\\
		&0 & 5 & 4 &\vline&3&\\
	\end{pmatrix}\\ \quad 
\end{equation*}
$\implies$ no, (-5,11,-7) $\notin$ span\{(1,-2,2),(0,5,5), (2,0,8)\}









\end{document}